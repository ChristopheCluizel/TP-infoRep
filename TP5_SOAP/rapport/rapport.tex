\documentclass{article}
\usepackage[utf8]{inputenc}
\usepackage[T1]{fontenc}
\usepackage{lmodern}
\usepackage[francais]{babel, varioref}
\usepackage{graphicx}
\usepackage{listings}

\usepackage{xspace}
\usepackage{amssymb}
\usepackage{calc}
\usepackage{listingsutf8}
\usepackage{color}
\usepackage{xcolor}

%%%%%%%%%%%%%%%%%%%%%
\usepackage{url}
\usepackage[top=0.1cm,bottom=0.1cm,left=0.6cm,right=0.2cm]{geometry}
\usepackage[final]{pdfpages}
%%%%%%%%%%%%%%%%%%%%%%%%%%%

%pour la coloration du code

\definecolor{colFond}{rgb}{0.8,0.9,0.9}
\definecolor{hellgelb}{rgb}{1,1,0.8}
\definecolor{colKeys}{rgb}{0,0,1}
\definecolor{colIdentifier}{rgb}{0,0,0}
\definecolor{colComments}{rgb}{0,0.5,0}
\definecolor{colString}{rgb}{0.62,0.12,0.94}

\lstset{
  language=JAVA,
  float=hbp,
  basicstyle=\ttfamily\small,
  identifierstyle=\color{colIdentifier},
  keywordstyle=\bf \color{colKeys},
  stringstyle=\color{colString},
  commentstyle=\color{colComments},
  columns=flexible,
  tabsize=3,
  frame=single,
  frame=shadowbox,
  rulesepcolor=\color[gray]{0.5},
  extendedchars=true,
  showspaces=false,
  showstringspaces=false,
  %numbers=left,
  firstnumber=1,
  numberstyle=\tiny,
  breaklines=true,
  backgroundcolor=\color{hellgelb},
  captionpos=b,
  basicstyle=\ttfamily\scriptsize
}

\title{\vspace{-5ex}SOAP\vspace{-2ex}}
\author{Alexandre \bsc{Brehmer} \& Christophe \bsc{Cluizel}}

\begin{document}
\maketitle
\vspace{-6ex}

TODO:
\begin{itemize}
  \item écrire l'interface côté serveur et sa classe concrète qui l'implémente
  \item déployer le serveur sous JBOSS (définir la variable JBOSS\_HOME) et le lancer (\textbf{JBOSS\_HOME/bin/standalone.sh})
  \item générer automatiquement les fichiers côté client, écrire le client et le lancer
\end{itemize}

Organisation des fichiers
\begin{lstlisting}
  + Client.
    - src
      * Client.java
      * calculatrice
        -> fichiers generes automatiquement
  + serveur.
    - Calculatrice.war
    - WEB-INF
      * web.xml
    - calculatrice
      * Calculatrice.java
      * CalculatriceWS.java
      * Complexe.java
  + compileClient.sh.
  + compileServeur.sh.
  + runClient.sh.
\end{lstlisting}

Compilation et déploiement du serveur.
\begin{lstlisting}[language=bash]
# Clean des repertoires
cd serveur; rm -Rf WEB-INF/classes; mkdir -p WEB-INF/classes
# Compilation et generation du WAR
javac -d WEB-INF/classes calculatrice/*.java; jar -cf Calculatrice.war WEB-INF;
# Deploiement
cp Calculatrice.war $JBOSS_HOME/standalone/deployments ;cd ..
\end{lstlisting}

Générer automatiquement les fichiers clients et les compiler
\begin{lstlisting}[language=bash]
# Clean des repertoires
cd client; rm -Rf calculatrice; mkdir calculatrice; cd src; rm -Rf calculatrice
# Recuperation du WSDL et generation des fichiers
$JBOSS_HOME/bin/wsconsume.sh -k -o . http://localhost:8080/Calculatrice/CalculatriceWS?wsdl
# Compilation du client
javac -d .. Client.java
mv calculatrice/*.class ../calculatrice; cd ../..
\end{lstlisting}

Exécuter le client.
\begin{lstlisting}[language=bash]
cd client; java Client $1 $2 $3 $4 $5; cd ..
\end{lstlisting}

Web.xml
\begin{lstlisting}[language=xml]
<?xml version="1.0" encoding="ISO-8859-1"?>
<!DOCTYPE web-app PUBLIC
"-//Sun Microsystems, Inc.//DTD Web Application 2.3//EN"
"http://java.sun.com/dtd/web-app_2_3.dtd">
<web-app>
  <servlet>
    <servlet-name>CalculatriceWS</servlet-name>
    <servlet-class>calculatrice.CalculatriceWS</servlet-class>
  </servlet>
  <servlet-mapping>
    <servlet-name>CalculatriceWS</servlet-name>
    <url-pattern>/CalculatriceWS</url-pattern>
  </servlet-mapping>
</web-app>
\end{lstlisting}

Interface serveur de la Calculatrice
\begin{lstlisting}[language=JAVA]
package calculatrice;
import javax.jws.WebService;
import javax.jws.WebMethod;
import javax.jws.WebParam;
import javax.jws.soap.SOAPBinding;

@WebService(name="Calculatrice")
@SOAPBinding(style=SOAPBinding.Style.RPC)
public interface Calculatrice {
  @WebMethod(action="addition")
  public Complexe addition(@WebParam(name="a") Complexe a, @WebParam(name="b") Complexe b);
  @WebMethod(action="soustraction")
  public Complexe soustraction(@WebParam(name="a") Complexe a, @WebParam(name="b") Complexe b);@WebMethod(action="multiplication")
  public Complexe multiplication(@WebParam(name="a") Complexe a, @WebParam(name="b") Complexe b);
  @WebMethod(action="division")
  public Complexe division(@WebParam(name="a") Complexe a, @WebParam(name="b") Complexe b);
}
\end{lstlisting}

Classe concrète serveur de la Calculatrice
\begin{lstlisting}[language=JAVA]
package calculatrice;
import javax.jws.WebService;

@WebService(endpointInterface="calculatrice.Calculatrice")
public class CalculatriceWS implements Calculatrice {
  @Override
  public Complexe addition(Complexe a, Complexe b)
  {
    return new Complexe(a.getPartieReelle() + b.getPartieReelle(), a.getPartieImaginaire() + b.getPartieImaginaire());
  }
  @Override
  public Complexe soustraction(Complexe a, Complexe b)
  {
    return new Complexe(a.getPartieReelle() - b.getPartieReelle(), a.getPartieImaginaire() - b.getPartieImaginaire());
  }
  @Override
  public Complexe division(Complexe a, Complexe b)
  {
    int reel = (a.getPartieReelle() * b.getPartieReelle() + a.getPartieImaginaire() * b.getPartieImaginaire()) / (b.getPartieReelle() * b.getPartieReelle() +
    int imaginaire = (a.getPartieReelle() * b.getPartieImaginaire() - b.getPartieReelle() * a.getPartieImaginaire()) / (b.getPartieReelle() * b.getPartieReel
    return new Complexe(reel, imaginaire);
  }
  @Override
  public Complexe multiplication(Complexe a, Complexe b)
  {
    return new Complexe(
    a.getPartieReelle() * b.getPartieReelle() - a.getPartieImaginaire() * b.getPartieImaginaire(),
    a.getPartieReelle() * b.getPartieImaginaire() + b.getPartieReelle() * a.getPartieImaginaire());
  }
}
\end{lstlisting}

Client de la calculatrice
\begin{lstlisting}[language=JAVA]
import calculatrice.*;

public class Client {
  public static void main(String[] args) throws java.lang.Exception {
    Calculatrice calculatrice = (new calculatrice.CalculatriceWSService()).getCalculatriceWSPort();
    int reelA = Integer.parseInt(args[0]);
    int imA = Integer.parseInt(args[1]);
    String op = args[2];
    int reelB = Integer.parseInt(args[3]);
    int imB = Integer.parseInt(args[4]);
    Complexe a = new Complexe();
    a.setReel(reelA);
    a.setImaginaire(imA);
    Complexe b = new Complexe();
    b.setReel(reelB);
    b.setImaginaire(imB);
    Complexe res = new Complexe();
    if (op.equals("+")) {
      res = calculatrice.addition(a, b);
      System.out.println(res.getReel() + " + " + res.getImaginaire() + "i");
    }
    if (op.equals("-")) {
      res = calculatrice.soustraction(a, b);
      System.out.println(res.getReel() + " + " + res.getImaginaire() + "i");
    }
    if (op.equals("mul")) {
      res = calculatrice.multiplication(a, b);
      System.out.println(res.getReel() + " + " + res.getImaginaire() + "i");
    }
    if (op.equals("div")) {
      res = calculatrice.division(a, b);
      System.out.println(res.getReel() + " + " + res.getImaginaire() + "i");
    }
  }
}
\end{lstlisting}

\end{document}
