\documentclass{article}
\usepackage[utf8]{inputenc}
\usepackage[T1]{fontenc}
\usepackage{lmodern}
\usepackage[francais]{babel, varioref}
\usepackage{graphicx}
\usepackage{listings}

\usepackage{xspace}
\usepackage{amssymb}
\usepackage{calc}
\usepackage{listingsutf8}
\usepackage{color}
\usepackage{xcolor}

%%%%%%%%%%%%%%%%%%%%%
\usepackage{url}
\usepackage[top=2.1cm,bottom=2.2cm,left=2cm,right=2cm]{geometry}
\usepackage[final]{pdfpages}
%%%%%%%%%%%%%%%%%%%%%%%%%%%

%pour la coloration du code

\definecolor{colFond}{rgb}{0.8,0.9,0.9}
\definecolor{hellgelb}{rgb}{1,1,0.8}
\definecolor{colKeys}{rgb}{0,0,1}
\definecolor{colIdentifier}{rgb}{0,0,0}
\definecolor{colComments}{rgb}{0,0.5,0}
\definecolor{colString}{rgb}{0.62,0.12,0.94}

\lstset{
  language=c,
  float=hbp,
  basicstyle=\ttfamily\small,
  identifierstyle=\color{colIdentifier},
  keywordstyle=\bf \color{colKeys},
  stringstyle=\color{colString},
  commentstyle=\color{colComments},
  columns=flexible,
  tabsize=3,
  frame=single,
  frame=shadowbox,
  rulesepcolor=\color[gray]{0.5},
  extendedchars=true,
  showspaces=false,
  showstringspaces=false,
  numbers=left,
  firstnumber=1,
  numberstyle=\tiny,
  breaklines=true,
  backgroundcolor=\color{hellgelb},
  captionpos=b,
}

\title{Info Rep TP1-Sockets}
\author{Alexandre \bsc{Brehmer} \& Christophe \bsc{Cluizel}}

\begin{document}
\maketitle
Client UDP Java
\begin{lstlisting}[language=JAVA]
import java.io.*;
import java.net.*;

public class ClientUDP{
  final static int taille = 1024;
  final static byte buffer[] = new byte[taille];
  public static void main(String args[]) throws Exception{
      String message = args[0] + args[1] + args[2] + "\n";  // creation du message
      InetAddress serveur = InetAddress.getByName("172.18.17.178"); // assignation de l'adresse ip du serveur
      int length = message.length();  // recupere la taille du message
      byte buffer[] = message.getBytes(); // place le message dans le buffer
      DatagramPacket dataSent = new DatagramPacket(buffer, length, serveur, 2000);  // creation du datagramme
      DatagramSocket socket = new DatagramSocket(); // creation du socket
      socket.send(dataSent);  // envoi des donnees
      DatagramPacket dataReceived = new DatagramPacket(new byte[100], 100); // 100 est le nombre de caracteres que l'on peut recevoir
      socket.receive(dataReceived); // reception des donnees = reponse du serveur
      System.out.println("Data received : " + new String(dataReceived.getData()));
      System.out.println("From : " + dataReceived.getAddress() + ":" + dataReceived.getPort());
    }
}
\end{lstlisting}

Serveur UDP C
\begin{lstlisting}[language=c]
#include <sys/socket.h>
#include <netinet/in.h>
#include <stdio.h>
#include <stdlib.h>
#include <string.h>
int main(int argc, char**argv){
    int sockfd,n;
    struct sockaddr_in servaddr,cliaddr;
    socklen_t len;
    char mesg[1000];
    sockfd=socket(AF_INET,SOCK_DGRAM,0);
    bzero(&servaddr,sizeof(servaddr));
    servaddr.sin_family = AF_INET;
    servaddr.sin_addr.s_addr=htonl(INADDR_ANY);
    servaddr.sin_port=htons(2000);
    bind(sockfd,(struct sockaddr *)&servaddr,sizeof(servaddr));
    for (;;) {
        len = sizeof(cliaddr);
        n = recvfrom(sockfd,mesg,1000,0,(struct sockaddr *)&cliaddr,&len);
        char res[100];
        sprintf(res,"%s","Bonjour \n\r");
        sendto(sockfd,res,sizeof(res),0,(struct sockaddr *)&cliaddr,sizeof(cliaddr));
        mesg[n] = 0;
    }
}
\end{lstlisting}
Client TCP Java
\begin{lstlisting}[language=JAVA]
import java.io.*;
import java.util.*;
import java.net.*;
public  class ClientTCP {
  public static void main(String[] args) throws Exception {
    Socket s;
    PrintStream socketOutputStream;
    BufferedReader socketInputStream;
    StringTokenizer tokens;
    int erreur; String reponse;
    s = new Socket("172.18.17.178", 2000);
    socketOutputStream = new PrintStream(s.getOutputStream());
    socketOutputStream.print("Bonjour\n");
    socketInputStream = new BufferedReader(new InputStreamReader(s.getInputStream()));
    if((reponse = socketInputStream.readLine()) != null)
      System.out.println("reponse:" + reponse);
    else
      System.out.println("Reponse vide !");
  }
}
\end{lstlisting}

Serveur TCP C
\begin{lstlisting}[language=c]
#include <sys/socket.h>
#include <netinet/in.h>
#include <stdio.h>

int main(int argc, char**argv){
   int listenfd,connfd,n;
   struct sockaddr_in servaddr,cliaddr;
   socklen_t clilen;
   pid_t     childpid;
   char mesg[1000];
   listenfd=socket(AF_INET,SOCK_STREAM,0);
   bzero(&servaddr,sizeof(servaddr));
   servaddr.sin_family = AF_INET;
   servaddr.sin_addr.s_addr=htonl(INADDR_ANY);
   servaddr.sin_port=htons(2000);
   bind(listenfd,(struct sockaddr *)&servaddr,sizeof(servaddr));//Creation de la socket
   listen(listenfd,2000);//ecoute sur le port 2000
   for(;;){
      clilen=sizeof(cliaddr);
      connfd = accept(listenfd,(struct sockaddr *)&cliaddr,&clilen);//acceptation de la connection
      if ((childpid = fork()) == 0){//creation d'un fils
         close (listenfd);
         for(;;){
            n = recvfrom(connfd,mesg,1000,0,(struct sockaddr *)&cliaddr,&clilen);
            mesg[n] = 0;
            strcpy(res,"Bonjour\n\0");
            sendto(connfd,res,sizeof(res),0,(struct sockaddr *)&cliaddr,sizeof(cliaddr));
         }
      }
      close(connfd);
   }
}
\end{lstlisting}


\end{document}
