\documentclass{article}
\usepackage[utf8]{inputenc}
\usepackage[T1]{fontenc}
\usepackage{lmodern}
\usepackage[francais]{babel, varioref}
\usepackage{graphicx}
\usepackage{listings}

\usepackage{xspace}
\usepackage{amssymb}
\usepackage{calc}
\usepackage{listingsutf8}
\usepackage{color}
\usepackage{xcolor}

%%%%%%%%%%%%%%%%%%%%%
\usepackage{url}
\usepackage[top=0.1cm,bottom=0.1cm,left=0.6cm,right=0.2cm]{geometry}
\usepackage[final]{pdfpages}
%%%%%%%%%%%%%%%%%%%%%%%%%%%

%pour la coloration du code

\definecolor{colFond}{rgb}{0.8,0.9,0.9}
\definecolor{hellgelb}{rgb}{1,1,0.8}
\definecolor{colKeys}{rgb}{0,0,1}
\definecolor{colIdentifier}{rgb}{0,0,0}
\definecolor{colComments}{rgb}{0,0.5,0}
\definecolor{colString}{rgb}{0.62,0.12,0.94}

\lstset{
  language=JAVA,
  float=hbp,
  basicstyle=\ttfamily\small,
  identifierstyle=\color{colIdentifier},
  keywordstyle=\bf \color{colKeys},
  stringstyle=\color{colString},
  commentstyle=\color{colComments},
  columns=flexible,
  tabsize=3,
  frame=single,
  frame=shadowbox,
  rulesepcolor=\color[gray]{0.5},
  extendedchars=true,
  showspaces=false,
  showstringspaces=false,
  %numbers=left,
  firstnumber=1,
  numberstyle=\tiny,
  breaklines=true,
  backgroundcolor=\color{hellgelb},
  captionpos=b,
}

\title{\vspace{-5ex}EJB\vspace{-2ex}}
\author{Alexandre \bsc{Brehmer} \& Christophe \bsc{Cluizel}}

\begin{document}
\maketitle
\vspace{-6ex}

TODO:
\begin{itemize}
	\item coder les interfaces Local et Remote.
	\item coder l'implémentation de ces interfaces.
	\item creer utilisateur JBOSS : \textbf{JBOSS\_HOME/bin/add\_user.sh}
	\item remplir \textbf{Client/jndi.properties} avec les infos de l'utilisateur cree.
	\item déployer le serveur.
	\item coder le client et l'executer.
\end{itemize}

Organisation des fichiers
\begin{lstlisting}
      + Calculatrice.
            - Calculatrice.java.
            - CalculatriceLocale.java.
            - CalculatriceRemote.java.
            - Complexe.java.
            - DivisionParZero.java.
      + Client.
            - Client.java.
            - jboss-ejb-client.properties.
            - jndi.properties.
       - compile.sh.
       - runClient.sh.
\end{lstlisting}

Compilation serveur et deploiement du serveur.
\begin{lstlisting}[language=bash]
export JBOSS_HOME=/tmp/jboss CLASSPATH=.:$JBOSS_HOME/modules/javax/ejb/api/main/jboss-ejb-api_3.1_spec-1.0.1.Final.jar 
# Compilation du serveur et de ses classes
javac -cp $CLASSPATH Calculatrice/*.java
# Deploiement du .jar
jar cvf Calculatrice.jar Calculatrice/*.class
cp Calculatrice.jar $JBOSS_HOME/standalone/deployments
# Copie des classes necessaires au client
cp Calculatrice/CalculatriceRemote.class Client/Calculatrice/
cp Calculatrice/Complexe.class Client/Calculatrice/
cp Calculatrice/DivisionParZero.class Client/Calculatrice/
# Compilation du client
javac Client/Client.java
\end{lstlisting}

Executer le client.
\begin{lstlisting}[language=bash]
export JBOSS_HOME=/tmp/jboss CLASSPATH=$JBOSS_HOME/bin/client/jboss-client.jar:. cd Client
java -cp $CLASSPATH Client $1 $2 $3 $4 $5
cd ..
\end{lstlisting}

jndi.properties
\begin{lstlisting}[language=bash]
java.naming.factory.initial=org.jboss.naming.remote.client.InitialContextFactory java.naming.provider.url=remote://127.0.0.1:4447
java.naming.security.principal=bob
java.naming.security.credentials=leponge
\end{lstlisting}

jbis-ejb-client.properties
\begin{lstlisting}[language=bash]
endpoint.name=client-endpoint
remote.connectionprovider.create.options.org.xnio.Options.SSL_ENABLED=false
remote.connections=default
remote.connection.default.host=127.0.0.1
remote.connection.default.port=4447
\end{lstlisting}


CalculatriceLocal.java
\begin{lstlisting}[language=JAVA]
package Calculatrice;
import java.rmi.Remote;
import java.rmi.RemoteException; import java.io.Serializable; import javax.ejb.Local;

@Local
public interface CalculatriceLocal extends Remote, Serializable {
public Complexe addition(Complexe a, Complexe b) throws RemoteException;
public Complexe soustraction(Complexe a, Complexe b) throws RemoteException;
public Complexe division(Complexe a, Complexe b) throws RemoteException, DivisionParZero;
public Complexe multiplication(Complexe a, Complexe b) throws RemoteException;
\end{lstlisting}

CalculatriceRemote.java
\begin{lstlisting}[language=JAVA]
package Calculatrice;
import java.rmi.RemoteException; import javax.ejb.Remote;

@Remote
public interface CalculatriceRemote {
public Complexe addition(Complexe a, Complexe b) throws RemoteException;
public Complexe soustraction(Complexe a, Complexe b) throws RemoteException;
public Complexe division(Complexe a, Complexe b) throws RemoteException, DivisionParZero;
public Complexe multiplication(Complexe a, Complexe b) throws RemoteException;
}
\end{lstlisting}

Calculatrice.java implementation Serveur
\begin{lstlisting}[language=JAVA]
package Calculatrice;
import java.rmi.RemoteException; import javax.ejb.Stateless;

@Stateless
public class Calculatrice implements CalculatriceLocal, CalculatriceRemote {
public Complexe addition(Complexe a, Complexe b) throws RemoteException {return a.plus(b);}
public Complexe soustraction(Complexe a, Complexe b) throws RemoteException {return a.minus(b);}
public Complexe division(Complexe a, Complexe b) throws RemoteException, DivisionParZero {`return a.divide(b);}
public Complexe multiplication(Complexe a, Complexe b) throws RemoteException {return a.times(b);}
}
\end{lstlisting}

Calculatrice.java implementation Client
\begin{lstlisting}[language=JAVA]
import javax.naming.InitialContext; import Calculatrice.CalculatriceRemote; import Calculatrice.Complexe;
public class Client {
  public static void main(String[] args) {
    try {
      CalculatriceRemote obj = (CalculatriceRemote)

      InitialContext.doLookup("Calculatrice/Calculatrice!Calculatrice.CalculatriceRemote");
      Complexe a = new Complexe(Integer.parseInt(args[0]), Integer.parseInt(args[1])); Complexe b = new Complexe(Integer.parseInt(args[3]), Integer.parseInt(args[4])); Complexe res = new Complexe(0, 0);
      if (args[2].equals("add"))
        res = obj.addition(a, b);
      if (args[2].equals("sub"))
        res = obj.soustraction(a, b);
      if (args[2].equals("div"))
        res = obj.division(a, b);
      if (args[2].equals("tim"))
        res = obj.multiplication(a, b);
      System.out.println("Resultat : " + res);
    } catch (Exception e) {
      System.out.println(e);
      e.printStackTrace();
      }
  }
}
\end{lstlisting}

\end{document}
