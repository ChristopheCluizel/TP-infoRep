\documentclass{article}
\usepackage[utf8]{inputenc}
\usepackage[T1]{fontenc}
\usepackage{lmodern}
\usepackage[francais]{babel, varioref}
\usepackage{graphicx}
\usepackage{listings}

\usepackage{xspace}
\usepackage{amssymb}
\usepackage{calc}
\usepackage{listingsutf8}
\usepackage{color}
\usepackage{xcolor}

%%%%%%%%%%%%%%%%%%%%%
\usepackage{url}
\usepackage[top=0.1cm,bottom=0.1cm,left=0.6cm,right=0.2cm]{geometry}
\usepackage[final]{pdfpages}
%%%%%%%%%%%%%%%%%%%%%%%%%%%

%pour la coloration du code

\definecolor{colFond}{rgb}{0.8,0.9,0.9}
\definecolor{hellgelb}{rgb}{1,1,0.8}
\definecolor{colKeys}{rgb}{0,0,1}
\definecolor{colIdentifier}{rgb}{0,0,0}
\definecolor{colComments}{rgb}{0,0.5,0}
\definecolor{colString}{rgb}{0.62,0.12,0.94}

\lstset{
  language=JAVA,
  float=hbp,
  basicstyle=\ttfamily\small,
  identifierstyle=\color{colIdentifier},
  keywordstyle=\bf \color{colKeys},
  stringstyle=\color{colString},
  commentstyle=\color{colComments},
  columns=flexible,
  tabsize=3,
  frame=single,
  frame=shadowbox,
  rulesepcolor=\color[gray]{0.5},
  extendedchars=true,
  showspaces=false,
  showstringspaces=false,
  %numbers=left,
  firstnumber=1,
  numberstyle=\tiny,
  breaklines=true,
  backgroundcolor=\color{hellgelb},
  captionpos=b,
}

\title{\vspace{-5ex}Info Rep TP4-RMI\vspace{-2ex}}
\author{Alexandre \bsc{Brehmer} \& Christophe \bsc{Cluizel}}

\begin{document}
\maketitle
\vspace{-6ex}
Réaliser une calculatrice distribuée pour nombres complexes à l'aide de RMI\@. Avec un serveur en Java, un service de nommage et un client en Java. Pour cela, il faut 4 dossiers: client, serveur, rmiregistry (service de nommage) et calculatriceComplexe. Il faut 5 fichiers: Client.java, Serveur.java, Calculatrice.java, CalculatriceImpl.java et Complexe.java. \\

Fichier \emph{compile.sh}
\vspace{-1.5ex}
\begin{lstlisting}[language=bash]
javac calculatriceComplexe/*.java
cp calculatriceComplexe/*.class serveur/calculatriceComplexe/

cp calculatriceComplexe/Calculatrice.class client/calculatriceComplexe/
cp calculatriceComplexe/Complexe.class client/calculatriceComplexe/

cp calculatriceComplexe/Calculatrice.class rmiregistry/calculatriceComplexe/
cp calculatriceComplexe/Complexe.class rmiregistry/calculatriceComplexe/

javac serveur/*.java
javac client/*.java
\end{lstlisting}
\vspace{-1ex}

Fichier \emph{Calculatrice.java} pour décrire les services proposés.
% \vspace{-1.5ex}
\begin{lstlisting}
package calculatriceComplexe;

import java.rmi.Remote;
import java.rmi.RemoteException;
import java.io.Serializable;

public interface Calculatrice extends Remote, Serializable {
  public Complexe addition(Complexe a, Complexe b) throws RemoteException;
  public Complexe soustraction(Complexe a, Complexe b) throws RemoteException;
  public Complexe multiplication(Complexe a, Complexe b) throws RemoteException;
  public Complexe division(Complexe a, Complexe b) throws RemoteException;
}
\end{lstlisting}

Fichier \emph{CalculatriceImpl.java} pour l'implémentation des services de la calculatrice.
% \vspace{-1.5ex}
\begin{lstlisting}
package calculatriceComplexe;

public class CalculatriceImpl implements Calculatrice {
  public Complexe addition(Complexe a, Complexe b) {
    return a.plus(b);
  }
  public Complexe soustraction(Complexe a, Complexe b) {
     return a.minus(b);
  }
  public Complexe multiplication(Complexe a, Complexe b) {
    return a.times(b);
  }
  public Complexe division(Complexe a, Complexe b) {
    return a.divides(b);
  }
}
\end{lstlisting}

Fichier \emph{Serveur.java}.
% \vspace{-1.5ex}
\begin{lstlisting}
import java.rmi.registry.LocateRegistry;
import java.rmi.registry.Registry;
import java.rmi.server.UnicastRemoteObject;
import java.util.Arrays;

import calculatriceComplexe.*;

public class Serveur {
  public static void main(String args[]) {
    int port  = 1099;

    if(args.length==1)
        port = Integer.parseInt(args[0]);

    try {
        Calculatrice stub = (Calculatrice)UnicastRemoteObject.exportObject(new CalculatriceImpl(), 0);
        Registry registry = LocateRegistry.getRegistry(port);

        if(!Arrays.asList(registry.list()).contains("CalculatriceComplexe"))
            registry.bind("CalculatriceComplexe", stub);
        else
            registry.rebind("CalculatriceComplexe", stub);
        System.out.println("Service CalculatriceComplexe lie au registre");

    } catch (Exception e) {
        System.out.println(e);
    }
  }
}
\end{lstlisting}

Fichier \emph{Client.java}.
% \vspace{-1.5ex}
\begin{lstlisting}
import java.rmi.registry.LocateRegistry;
import java.rmi.registry.Registry;

import calculatriceComplexe.*;

public class Client {
  public static void main(String args[]) {
    String machine = "localhost";
    int port = 1099;

    try {
    Registry registry = LocateRegistry.getRegistry(machine, port);
    Calculatrice obj = (Calculatrice)registry.lookup("CalculatriceComplexe");

    Complexe a = new Complexe(Integer.parseInt(args[0]), Integer.parseInt(args[1]));
    Complexe b = new Complexe(Integer.parseInt(args[3]), Integer.parseInt(args[4]));
    String operateur = args[2];

    if(operateur.contains("+"))
      System.out.println(obj.addition(a, b));
    if(operateur.contains("-"))
      System.out.println(obj.soustraction(a, b));
    if(operateur.contains("x"))
      System.out.println(obj.multiplication(a, b));
    if(operateur.contains("/"))
      System.out.println(obj.division(a, b));

    } catch (Exception e) {
    System.out.println("Client exception: " + e);
    }
  }
}
\end{lstlisting}

Fichier \emph{Complexe.java} pour l'implémentation des nombres complexes.
% \vspace{-1.5ex}
\begin{lstlisting}
package calculatriceComplexe;

import java.io.Serializable;

public class Complexe implements Serializable{
    private final double re;   // the real part
    private final double im;   // the imaginary part

    public Complexe(double real, double imag) {...}
    public String toString() {...}
    public Complexe plus(Complexe b) {...}
    public Complexe minus(Complexe b) {...}
    public Complexe times(Complexe b) {...}
    public Complexe reciprocal() {...}
    public double re() { return re; }
    public double im() { return im; }
    public Complexe divides(Complexe b) {...}
}

\end{lstlisting}

\end{document}
