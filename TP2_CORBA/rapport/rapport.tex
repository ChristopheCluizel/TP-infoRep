\documentclass{article}
\usepackage[utf8]{inputenc}
\usepackage[T1]{fontenc}
\usepackage{lmodern}
\usepackage[francais]{babel, varioref}
\usepackage{graphicx}
\usepackage{listings}

\usepackage{xspace}
\usepackage{amssymb}
\usepackage{calc}
\usepackage{listingsutf8}
\usepackage{color}
\usepackage{xcolor}

%%%%%%%%%%%%%%%%%%%%%
\usepackage{url}
\usepackage[top=2.1cm,bottom=2.2cm,left=2cm,right=2cm]{geometry}
\usepackage[final]{pdfpages}
%%%%%%%%%%%%%%%%%%%%%%%%%%%

%pour la coloration du code

\definecolor{colFond}{rgb}{0.8,0.9,0.9}
\definecolor{hellgelb}{rgb}{1,1,0.8}
\definecolor{colKeys}{rgb}{0,0,1}
\definecolor{colIdentifier}{rgb}{0,0,0}
\definecolor{colComments}{rgb}{0,0.5,0}
\definecolor{colString}{rgb}{0.62,0.12,0.94}

\lstset{
  language=c,
  float=hbp,
  basicstyle=\ttfamily\small,
  identifierstyle=\color{colIdentifier},
  keywordstyle=\bf \color{colKeys},
  stringstyle=\color{colString},
  commentstyle=\color{colComments},
  columns=flexible,
  tabsize=3,
  frame=single,
  frame=shadowbox,
  rulesepcolor=\color[gray]{0.5},
  extendedchars=true,
  showspaces=false,
  showstringspaces=false,
  numbers=left,
  firstnumber=1,
  numberstyle=\tiny,
  breaklines=true,
  backgroundcolor=\color{hellgelb},
  captionpos=b,
}

\title{Info Rep TP2-Corba}
\author{Alexandre \bsc{Brehmer} \& Christophe \bsc{Cluizel}}

\begin{document}
\maketitle
Corba est une librairie multi-langage permettant de simplifier l'appel de fonction à distance ainsi que la synchronisation des objets. Sont fonctionnement est décrit dans 4 fichiers :
\begin{itemize}
\item client.java\\
	Il fait appel aux fonctions en se connectant au serveur.
\item serveur.java\\
	Il met à disposition les fonctions en démarrant Corba.
\item application.idl\\
	C'est ici qu'un déclaration des fonctions est faite pour les rendre accessibles à tout langages de programmation.
\item servant.java\\
	C'est ici que les fonctions sont implémentées
\end{itemize}

helloClient.java
\begin{lstlisting}[language=JAVA]
import HelloWorld.*;
import org.omg.CORBA.*;
public class HelloClient {
  public static void main(String args[]) {
    try {
      String orbArguments[] = new String[0];
      ORB orb = ORB.init(orbArguments, null);
      org.omg.CORBA.Object corbaObject = orb.string_to_object(args[0]);
      Hello corbaHelloObject = HelloHelper.narrow(corbaObject);
       System.out.println(corbaHelloObject.sayHello(""));
        if(args[2].contains("+"))
		System.out.println(corbaHelloObject.addition(Integer.parseInt(args[1]),Integer.parseInt(args[3])));
    }
    catch (Exception e) {
      System.err.println("Error : "+e);
    }
  }
}
\end{lstlisting}

helloServeur.java
\begin{lstlisting}[language=JAVA]
import HelloWorld.*;
import org.omg.CORBA.*;
import org.omg.PortableServer.*;
import org.omg.PortableServer.POA;
public class HelloServer {
  public static void main(String args[]) {
    try {
      String orbArguments[]=new String[0];
      ORB orb=ORB.init(orbArguments,null);
      POA rootpoa = POAHelper.narrow(orb.resolve_initial_references("RootPOA"));
      rootpoa.the_POAManager().activate();
      HelloServant helloServant=new HelloServant();
      Hello corbaHelloObject = HelloHelper.narrow(rootpoa.servant_to_reference(helloServant));
      System.out.println(orb.object_to_string(corbaHelloObject));
      orb.run();
    }
    catch (Exception e) {
      System.err.println("Error : "+e);
    }
  }
}
\end{lstlisting}

helloServant.java
\begin{lstlisting}[language=JAVA]
import HelloWorld.*;
import HelloWorld.HelloPackage.*;
class HelloServant extends HelloPOA {
  public String sayHello(String name) throws ChaineVide {
    if(name.equals(""))
	throw new ChaineVide();
    System.out.println("Request from " + name);
    return "Hello "+name;
  }
  public int addition(int a,int b) throws Negatif {
	  if(a<0 || b<0) throw new Negatif();
	  return a+b;
  }  
  public int soustraction(int a,int b) throws Negatif {
	  if(a<0 || b<0 || a-b<0) throw new Negatif();
	  return a-b;
  }  
  public int multiplication(int a,int b) throws Negatif {
	  if(a<0 || b<0) throw new Negatif();
	  return a*b;
  }  
  public int division(int a,int b) throws Negatif, DivisionParZero  {
	  if(a<0 || b<0) throw new Negatif();
	  if(b==0) throw new DivisionParZero();
	  return a/b;
  }  
}


\end{lstlisting}

helloWorld.idl
\begin{lstlisting}[language=JAVA]
module HelloWorld {
  interface Hello {
    exception ChaineVide { };
    exception Negatif { };
    exception DivisionParZero { };
    string sayHello(in string name) raises (ChaineVide);
    unsigned long addition(in unsigned long a, in unsigned long b) raises (Negatif);
    unsigned long soustraction(in unsigned long a, in unsigned long b) raises (Negatif);
    unsigned long multiplication(in unsigned long a, in unsigned long b) raises (Negatif);
    unsigned long division(in unsigned long a, in unsigned long b) raises (Negatif, DivisionParZero);
  };
};
\end{lstlisting}

La compilation s'effectue avec la commande:
\begin{lstlisting}
QU'EST CE QUE C'EST LA COMMANDE?
\end{lstlisting}

\end{document}
