\documentclass{article}
\usepackage[utf8]{inputenc}
\usepackage[T1]{fontenc}
\usepackage{lmodern}
\usepackage[francais]{babel, varioref}
\usepackage{graphicx}
\usepackage{listings}

\usepackage{xspace}
\usepackage{amssymb}
\usepackage{calc}
\usepackage{listingsutf8}
\usepackage{color}
\usepackage{xcolor}

%%%%%%%%%%%%%%%%%%%%%
\usepackage{url}
\usepackage[top=0.1cm,bottom=0.1cm,left=0.6cm,right=0.2cm]{geometry}
\usepackage[final]{pdfpages}
%%%%%%%%%%%%%%%%%%%%%%%%%%%

%pour la coloration du code

\definecolor{colFond}{rgb}{0.8,0.9,0.9}
\definecolor{hellgelb}{rgb}{1,1,0.8}
\definecolor{colKeys}{rgb}{0,0,1}
\definecolor{colIdentifier}{rgb}{0,0,0}
\definecolor{colComments}{rgb}{0,0.5,0}
\definecolor{colString}{rgb}{0.62,0.12,0.94}

\lstset{
  language=c,
  float=hbp,
  basicstyle=\ttfamily\small,
  identifierstyle=\color{colIdentifier},
  keywordstyle=\bf \color{colKeys},
  stringstyle=\color{colString},
  commentstyle=\color{colComments},
  columns=flexible,
  tabsize=3,
  frame=single,
  frame=shadowbox,
  rulesepcolor=\color[gray]{0.5},
  extendedchars=true,
  showspaces=false,
  showstringspaces=false,
  numbers=left,
  firstnumber=1,
  numberstyle=\tiny,
  breaklines=true,
  backgroundcolor=\color{hellgelb},
  captionpos=b,
}

\title{\vspace{-5ex}Info Rep TP3-Corba2\vspace{-2ex}}
\author{Alexandre \bsc{Brehmer} \& Christophe \bsc{Cluizel}\& Jean Claude \bsc{Bernard}\vspace{-2ex}}

\begin{document}
\maketitle
\vspace{-6ex}
Réaliser une calculatrice distribuée sur nombres complexes à l'aide de Corba. Avec un serveur en C, un service de nommage et un client en JAVA.


CalculatriceComplexes.idl
\vspace{-1.5ex}
\begin{lstlisting}[language=JAVA]
module CalculatriceComplexeCorba {
 	struct Complexe { double re; double im; };
	interface Calculatrice{ exception CalculatriceErreur { string raison;  };
		Complexe additionner(in Complexe c1, in Complexe c2);
		Complexe soustraire(in Complexe c1, in Complexe c2);
		Complexe multiplier(in Complexe c1, in Complexe c2);
		Complexe diviser(in Complexe c1, in Complexe c2) raises (CalculatriceErreur);	}; };
\end{lstlisting}


\vspace{-1ex}
CompileIDL.sh
\vspace{-1.5ex}
\begin{lstlisting}
orbit-idl-2 --skeleton-impl --output-dir=Serveur CalculatriceComplexes.idl
idlj -fclient -keep -td Client CalculatriceComplexes.idl
\end{lstlisting}


\vspace{-1ex}
CalculatriceComplexes-skelimpl.c
\vspace{-1.5ex}
\begin{lstlisting}[language=c]
/* This is a template file generated by command  orbit-idl-2 --skeleton-impl CalculatriceComplexes.idl */
#include "CalculatriceComplexes.h" #include "math/Complexe.h"
...
#if !defined(_impl_CalculatriceComplexeCorba_Calculatrice_additionner_)
#define _impl_CalculatriceComplexeCorba_Calculatrice_additionner_ 1
static CalculatriceComplexeCorba_Complexe impl_CalculatriceComplexeCorba_Calculatrice_additionner(impl_POA_CalculatriceComplexeCorba_Calculatrice *servant, const CalculatriceComplexeCorba_Complexe* c1,
const CalculatriceComplexeCorba_Complexe* c2, CORBA_Environment *ev) {
CalculatriceComplexeCorba_Complexe retval;
Complexe z1 = complexe_creer(c1->re, c1->im), z2 = complexe_creer(c2->re, c2->im), z;
   z = complexe_additionner(z1, z2);
   retval.re = z.re; retval.im = z.im; return retval; }
#endif
#if !defined(_impl_CalculatriceComplexeCorba_Calculatrice_soustraire_)
#define _impl_CalculatriceComplexeCorba_Calculatrice_soustraire_ 1
static CalculatriceComplexeCorba_Complexe impl_CalculatriceComplexeCorba_Calculatrice_soustraire(impl_POA_CalculatriceComplexeCorba_Calculatrice *servant, const CalculatriceComplexeCorba_Complexe* c1,
const CalculatriceComplexeCorba_Complexe* c2, CORBA_Environment *ev) {
CalculatriceComplexeCorba_Complexe retval;
   Complexe z1 = complexe_creer(c1->re, c1->im), z2 = complexe_creer(c2->re, c2->im), z;
   z = complexe_soustraire(z1, z2);
   retval.re = z.re;    retval.im = z.im; return retval; }
#endif
#if !defined(_impl_CalculatriceComplexeCorba_Calculatrice_multiplier_)
#define _impl_CalculatriceComplexeCorba_Calculatrice_multiplier_ 1
static CalculatriceComplexeCorba_Complexe impl_CalculatriceComplexeCorba_Calculatrice_multiplier(impl_POA_CalculatriceComplexeCorba_Calculatrice *servant, const CalculatriceComplexeCorba_Complexe* c1,
const CalculatriceComplexeCorba_Complexe* c2, CORBA_Environment *ev) {
CalculatriceComplexeCorba_Complexe retval;
   Complexe z1 = complexe_creer(c1->re, c1->im), z2 = complexe_creer(c2->re, c2->im), z;
   z = complexe_multiplier(z1, z2);
   retval.re = z.re;    retval.im = z.im; return retval; }
#endif
#if !defined(_impl_CalculatriceComplexeCorba_Calculatrice_diviser_)
#define _impl_CalculatriceComplexeCorba_Calculatrice_diviser_ 1
static CalculatriceComplexeCorba_Complexe impl_CalculatriceComplexeCorba_Calculatrice_diviser(impl_POA_CalculatriceComplexeCorba_Calculatrice *servant, const CalculatriceComplexeCorba_Complexe* c1,
const CalculatriceComplexeCorba_Complexe* c2, CORBA_Environment *ev) {
CalculatriceComplexeCorba_Complexe retval;
Complexe z1 = complexe_creer(c1->re, c1->im), z2 = complexe_creer(c2->re, c2->im), z; int erreur;
CalculatriceComplexeCorba_Calculatrice_CalculatriceErreur* exception;
erreur = complexe_diviser(z1, z2, &z);
if (erreur==0) { retval.re = z.re; retval.im = z.im; }
else {     exception = CalculatriceComplexeCorba_Calculatrice_CalculatriceErreur__alloc();
     exception->raison = CORBA_string_dup ("Division par 0.");
     CORBA_exception_set(ev, CORBA_USER_EXCEPTION, ex_CalculatriceComplexeCorba_Calculatrice_CalculatriceErreur,
exception); }
return retval; }
#endif
\end{lstlisting}


\vspace{-1ex}
serveur.c
\vspace{-1.5ex}
\begin{lstlisting}[language=c]
#include <stdio.h> #include <ORBitservices/CosNaming.h> #include <ORBitservices/CosNaming_impl.h>
#include "CalculatriceComplexes.h" #include "CalculatriceComplexes-skelimpl.c"
int main (int argc, char *argv[]) {
  PortableServer_POA rootpoa;
  CalculatriceComplexeCorba_Calculatrice corbaCalculatriceObject;
  CORBA_ORB orb=NULL;
  CORBA_Environment env;
  CosNaming_NamingContext nameServiceObject;
  CosNaming_NameComponent namePath[1] = {"CalculatriceComplexeCorba", ""};
  CosNaming_Name name = {1, 1, namePath, CORBA_FALSE};
  CORBA_exception_init(&env);
  orb = CORBA_ORB_init(&argc, argv, "orbit-local-orb", &env);
rootpoa = (PortableServer_POA)CORBA_ORB_resolve_initial_references(orb, "RootPOA", &env);
PortableServer_POAManager_activate(PortableServer_POA__get_the_POAManager(rootpoa, &env), &env);
  nameServiceObject = CORBA_ORB_resolve_initial_references(orb, "NameService", &env);
  corbaCalculatriceObject = impl_CalculatriceComplexeCorba_Calculatrice__create(rootpoa,&env);
  CosNaming_NamingContext_rebind(nameServiceObject, &name, corbaCalculatriceObject, &env);
  CORBA_ORB_run(orb, &env);
  CORBA_Object_release(corbaCalculatriceObject, &env);
  CORBA_ORB_shutdown(orb, CORBA_FALSE, &env);
return 0; }
\end{lstlisting}


\vspace{-1ex}
compileServeur.sh
\vspace{-1.5ex}
\begin{lstlisting}[language=c]
gcc -c math/Complexe.c
gcc -c CalculatriceComplexes-common.c 'orbit2-config --cflags --use-service=name server'
gcc -c CalculatriceComplexes-stubs.c 'orbit2-config --cflags --use-service=name server'
gcc -c CalculatriceComplexes-skels.c 'orbit2-config --cflags --use-service=name server'
gcc -c CalculatriceComplexes-skelimpl.c 'orbit2-config --cflags --use-service=name server'
gcc -c serveur.c 'orbit2-config --cflags --use-service=name server'
gcc -o Serveur Complexe.o CalculatriceComplexes-common.o CalculatriceComplexes-stubs.o CalculatriceComplexes-
     skels.o CalculatriceComplexes-skelimpl.o serveur.o 'orbit2-config --libs --use-service=name server'
\end{lstlisting}


\vspace{-1ex}
CalculatriceSurComplexeCORBA.java
\vspace{-1.5ex}
\begin{lstlisting}[language=JAVA]
import java.io.*; import java.util.*; import java.net.*; import org.omg.CORBA.*;
import org.omg.CosNaming.*; import CalculatriceComplexeCorba.*; import CalculatriceComplexeCorba.CalculatricePackage.*;
public class CalculatriceSurComplexeCORBA {
  CalculatriceComplexeCorba.Calculatrice calculatriceComplexeCORBA;
public CalculatriceSurComplexeCORBA(String[] args) throws Exception { ORB orb = ORB.init(args, null);
NamingContext corbaNamingServiceReference = NamingContextHelper.narrow(orb.resolve_initial_references("NameService"));
NameComponent calculatriceName = new NameComponent("CalculatriceComplexeCorba", ""); NameComponent calculatricePath[] = {calculatriceName};
this.calculatriceComplexeCORBA=CalculatriceComplexeCorba.CalculatriceHelper.narrow(   corbaNamingServiceReference.resolve(calculatricePath));}
public Complexe additionner(Complexe c1,Complexe c2) throws Exception{return calculatriceComplexeCORBA.additionner(c1, c2);}
public Complexe soustraire(Complexe c1, Complexe c2) throws Exception{return calculatriceComplexeCORBA.soustraire(c1, c2); }
public Complexe multiplier(Complexe c1, Complexe c2) throws Exception{return calculatriceComplexeCORBA.multiplier(c1, c2); }
public Complexe diviser(Complexe c1, Complexe c2) throws Exception { return calculatriceComplexeCORBA.diviser(c1, c2); } }
\end{lstlisting}


\vspace{-1ex}
Client.java
\vspace{-1.5ex}
\begin{lstlisting}[language=JAVA]
public class Client {
public static void main(String[] args) throws Exception {
CalculatriceSurComplexeCORBA calc = new CalculatriceSurComplexeCORBA(args[0].split(" ")); CalculatriceComplexeCorba.Complexe un = new CalculatriceComplexeCorba.Complexe(1.0, 0.0);
zero = new CalculatriceComplexeCorba.Complexe(0.0, 0.0);
un_deux = new CalculatriceComplexeCorba.Complexe(1.0, 2.0); System.out.println("1 + (1+2i) = " + calc.additionner(un,un_deux));
    System.out.println("1 - (1+2i) = " + calc.soustraire(un,un_deux));
    System.out.println("1 x (1+2i) = " + calc.multiplier(un_deux,un_deux));
    System.out.println("(1+2i) / 1 = " + calc.diviser(un_deux,un));
    System.out.println("(1+2i) / (1+2i) = " + calc.diviser(un_deux,un_deux));
    System.out.println("1 / (1+2i) = " + calc.diviser(un,un_deux));
    System.out.println("(1+2i) / 0 = " + calc.diviser(un_deux,zero));  } }
\end{lstlisting}


\vspace{-1ex}
nommage.sh
\vspace{-1.5ex}
\begin{lstlisting}[language=c]
orbit-name-server-2 -ORBCorbaloc=1 -ORBIIOPIPv4=1 -ORBIIOPIPName=$1 -ORBIIOPIPSock=$2 --key=NamingService
\end{lstlisting}

Remarque : Ne pas oublier l’option "-ORBIIOPIPv4=1" quand on lance le serveur en C...

\end{document}
